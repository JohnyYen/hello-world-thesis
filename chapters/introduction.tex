\chapter*{Introducción}
\addcontentsline{toc}{chapter}{Introducción}

En el contexto del siglo XXI, uno de los desafíos más persistentes en el ámbito educativo es la necesidad de adaptar las metodologías de enseñanza a las demandas de una sociedad en constante evolución. A pesar de los avances tecnológicos y pedagógicos, la educación enfrenta retos significativos en todos los niveles, particularmente en la enseñanza universitaria. 

Los videojuegos, como fenómeno cultural y social, han adquirido un papel central en la vida de millones de jovenes a nivel global. Estos sistemas interactivos, caracterizados por la presencia de objetivos claros y un conjunto de elementos que guían al usuario hacia su consecución, han demostrado ser herramientas poderosas para captar la atención y fomentar la participación activa \cite{cerezo-pizarro_cultural_2023}. En este sentido, el juego se ha convertido en una actividad intrínsecamente atractiva para los jóvenes, tal como lo destacó Johan Huizinga en su obra \cite{Huizinga1950}, donde define el juego como “una actividad u ocupación voluntaria, ejercida dentro de ciertos límites de tiempo y espacio, que sigue reglas libremente aceptadas pero absolutamente obligatorias, y que va acompañada de un sentimiento de tensión y alegría, así como de una conciencia sobre su diferencia con la vida cotidiana”. 

La capacidad de los videojuegos para mantener la atención de los usuarios durante períodos de tiempo prolongados, junto con su potencial para desarrollar habilidades como el razonamiento lógico, la resolución de problemas y el trabajo colaborativo, los posiciona como recursos valiosos en el ámbito educativo \cite{cerezo-pizarro_cultural_2023}.

En este contexto, la gamificación emerge como una estrategia innovadora \cite{ortiz-colon_gamificacion_2018}. Según Marczewski (2015), en su obra Even Ninja Monkeys Like to Play, la gamificación se define como “el uso de ideas y elementos propios de los juegos en contextos ajenos a ellos, como el trabajo o la vida cotidiana”\cite{ortiz-colon_gamificacion_2018}. Este enfoque ha sido ampliamente adoptado en la educación, donde se ha demostrado su eficacia para incrementar la motivación intrínseca de los estudiantes y mejorar su compromiso con el proceso de aprendizaje.

La literatura especializada identifica los elementos de gamificación más recurrentes y efectivos: puntos, medallas y tablas de clasificación (leader-boards). Estudios como los de \cite{ortiz-colon_gamificacion_2018} y \cite{oliveira_does_2021} respaldan la eficacia de estas mecánicas, destacando su capacidad para motivar cambios conductuales positivos en los estudiantes, quienes se ven incentivados a competir y superar desafíos para obtener recompensas. Es importante destacar que la gamificación no se limita al uso de videojuegos como tal, sino que se enfoca en aprovechar los elementos y dinámicas que estos incorporan. En el marco de esta investigación, para el desarrollo del proyecto, se tomará como base su teoría, los elementos que la componen y el valor que otorga a la motivación de los estudiantes.

La efectividad de la gamificación en educación no depende únicamente de su diseño
lúdico, sino de su capacidad para integrarse con principios pedagógicosfundamentales. En este sentido, teorías como el constructivismo, el aprendizaje basado en problemas y la teoría del flujo ofrecen un sustento teórico clave:
\begin{itemize}
    \item El constructivismo enfatiza la importancia de actividades que promuevan la
exploración, la experimentación y el aprendizaje autónomo, facilitando así la
construcción de conocimiento de forma personalizada y significativa \cite{kiili_design_2012}. 

    \item El aprendizaje basado en problemas (ABP) enfoque fomenta un aprendizaje profundo y
significativo al vincular la teoría con la práctica de manera
interactiva \cite{gallego_panoramica_nodate}.

    \item La teoría del flujo explica cómo lograr inmersión y compromiso en actividades
educativas. El estado de flujo ocurre cuando hay equilibrio entre el desafío y las
habilidades del individuo, evitando aburrimiento (desafío bajo) o ansiedad (desafío
alto). Este balance es clave para diseñar experiencias educativas motivadoras y efectivas \cite{kiili_design_2012, bozkurt_systematic_2018}.
\end{itemize}

En el ámbito específico de la enseñanza de la programación, se han desarrollado
diversas herramientas y videojuegos educativos que buscan facilitar el aprendizaje de conceptos complejos:

\begin{itemize}
    \item GhostCoder: Es un videojuego serio diseñado para que los programadores novatos repasen los conceptos fundamentales de la programación. Consiste en un juego de cartas donde el jugador debe luchar con la computadora completando las tareas de programación que aparecen en las cartas. Basandonse en el rendimiento del jugador, el juego puede variar el número de opciones para usar o la complejidad del algoritmo a completar, manteniendo al jugador en un estado de \textit{flow} \cite{shum_personalised_2023}.
    
    \item Human Resource Machine (HRM): Es un videojuego educativo que enseña
programación básica mediante puzzles visuales donde el jugador escribe secuencias similares a código ensamblador, introduciendo conceptos básicos de la programación [11]; sin embargo, presenta limitaciones significativas al no ofrecer un seguimiento por parte de los docentes, se evalúa el cumplimiento de un contenido mediante la superación de niveles sin alguna validación extra, mantener una dificultad rígida sin adaptarse al ritmo individual y carecer de retroalimentación pedagógica ante errores. Estas limitaciones contradicen los principios del constructivismo al no fomentar una comprensión profunda mediante reflexión guiada, del aprendizaje basado en problemas al no promover estrategias de resolución autónoma con apoyo y de la teoría del flujo al no equilibrar desafío-habilidad de forma dinámica.

    \item Scratch: Es una plataforma de programación visual del MIT, introduce a los principiantes en el pensamiento computacional mediante proyectos creativos, pero muestra importantes limitaciones: su diseño no incorpora mecanismos para el monitoreo docente ni evaluaciones que validen la comprensión conceptual más allá de la ejecución técnica, carece de progresión adaptativa que ajuste los desafíos al desarrollo individual de competencias, y ofrece retroalimentación básica centrada en errores sintácticos más que en el fortalecimiento de habilidades lógicas [12]. Estas deficiencias evidencian un desalineamiento tanto con la teoría del flujo y del aprendizaje basado en problemas al depender de una construcción externa de la narrativa o de los niveles, en vez de utilizar un sistema nativo.
\end{itemize}

A partir de lo enunciado anteriormente, se identifica la siguiente \textbf{situación
problemática}:
Las soluciones existentes para el aprendizaje de programación, si bien son soluciones útiles, presentan limitaciones significativas. Entre estas, destaca la falta de un sistema que evalúe de manera efectiva el cumplimiento de los objetivos de aprendizaje en cada nivel, así como la ausencia de una retroalimentación clara y constructiva que guíe a los estudiantes en su proceso. Además, estas herramientas no proporcionan a los docentes un control adecuado para monitorear y apoyar el progreso de sus alumnos. Por otro lado, se observa una carencia en la adaptación de los niveles de dificultad, lo que impide que los estudiantes avancen de manera escalonada y acorde a su ritmo de aprendizaje. Estas deficiencias generan un vacío en la experiencia educativa, limitando tanto el desarrollo óptimo de los estudiantes como la capacidad de los profesores para intervenir de manera efectiva en su formación.

Las variables de investigación se definen a continuación: La \textbf{Variable Independiente} es el desarrollo de la plataforma educativa definida con sus módulos. Las \textbf{Variables Dependientes} son (1) el dominio de los temas tratados medido en una prueba pre/post y (2) el nivel de motivación medido con el uso de la plataforma.

Que se deriva en el presente \textbf{problema de investigación}:
¿Como diseñar una plataforma educativa basado en videojuegos donde el aprendizaje sea adaptativo y constructivo, que integre un aprendizaje mediante videojuegos  y permita monitorear el progreso de los estudiantes, desde la perspectiva del profesor?

A raíz de este problema se propone el siguiente \textbf{objetivo general}:
Desarrollar una plataforma educativa basada en mecánicas de videojuegos adaptativos para la enseñanza de la programación, que permita ajustar los niveles de dificultad al rendimiento del estudiante y proporcione a los docentes herramientas de seguimiento y evaluación continua, con el fin de fortalecer el proceso de enseñanza-aprendizaje en entornos universitarios.


Para su cumplimiento se desglosa en los siguientes \textbf{objetivos específicos} y \textbf{tareas de investigación}:

\begin{enumerate}
    \item Identificar las bases conceptuales, pedagógicas y psicologicas que prevalecen en el uso de videojuegos eduactivos para la enseñanza de la programación.
    \begin{enumerate}
        \item Sintetizar los elementos esenciales de los videojuegos, como medio cultural-social del siglo XXI.
        \item Identificar los principales videojuegos educativos de programación.
        \item Identificar las principales teorías del aprendizaje que rigen el desarrollo de videojuegos adaptativos.
    \end{enumerate}
    
    \item Analizar metodologías de diseño de videojuegos y seleccionar la que mejor se adapte al contexto educativo de programación
    \begin{enumerate}
        \item Analizar y comparar las metodologías de diseño de videojuegos existentes.
        \item Mapear conceptos de programación a elemento del videojuego.
        \item Escribir una narrativa coherente con el objetivo planteado
    \end{enumerate}

    \item Diseñar la arquitectura de la plataforma
    \begin{enumerate}
        \item Capturar los requisitos funcionales y no funcionales asociados a la plataforma.
        \item Diseñar los diagramas de caso de uso del sistema, de base de datos y de arquitectura de la plataforma.
        \item Justificar los principios y patrones de diseño esenciales para el desarrollo de la plataforma.
        \item Diseñar el modelo de despliegue de la plataforma
        \item Seleccionar las tecnologías utilizadas para el desarrollo de la plataforma.
    \end{enumerate}

    \item Validar la calidad técnica y usabilidad del videojuego
    \begin{enumerate}
        \item Realizar pruebas unitarias y de integración en la plataforma.
        \item Realizar pruebas de rendimiento del software.
        \item Realizar pruebas de conexión entre los módulos.
    \end{enumerate}

\end{enumerate}

El \textbf{Valor Agregado} de este proyecto radica en la implementación de una plataforma educativa que incorpora un videojuego educativo, que ajusta la dificultad de los niveles según el rendimiento de los estudiantes y generá retroalimentación personalizada para cada jugador. Asimismo, ofrece a los docentes herramientas de evaluación continua, lo que fortalece su capacidad para orientar el proceso de aprendizaje de manera personalizada y efectiva.

El presente documento está estructurado en 3 capitulos orientados a dar respuesta a cada uno de los objetivos específicos planteados:

\begin{enumerate}
    \item Capitulo 1. \chapterOne: Revisa las bases fundamentales y los antecedentes de los videojuegos eduactivos asi como su papel en la sociedad moderna y como estos influyen en el aprendizaje de los jovenes. Se analizan teorías relacionadas con el aprendizaje y como estas se pueden utilizar para diseñar un videojuego adaptativo para el aprendizaje de la programación.

    \item Capitulo 2. \chapterSecond: Se detalla el enfoque adoptado para el diseño de la solución requerida en la implementación de la plataforma. Se identifican los requisitos funcionales y no funcionales de la plataforma. Se diseñan los artefactos UML necesarios incluyendo el diagrama de caso de uso del sistema, de arquitectura y de despligue ,y modelo de base de datos.
    
    \item Capitulo 3. \chapterThree Presenta las pruebas técnicas realizadas a los componentes esenciales de la plataforma, así como una interpretación de los resultados obtenidos.
\end{enumerate}





