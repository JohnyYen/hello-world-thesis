\chapter*{Introducción}
\addcontentsline{toc}{chapter}{Introducción}

En el contexto del siglo XXI, uno de los desafíos más persistentes en el ámbito
educativo fue la necesidad de adaptar las metodologías de enseñanza a las demandas
de una sociedad en constante evolución. A pesar de los avances tecnológicos y
pedagógicos, la educación enfrenta retos significativos en todos los niveles escolares, particularmente en la enseñanza universitaria [1].

Los videojuegos, como fenómeno cultural y social, han adquirido un papel central en la vida de millones de jóvenes a nivel global. Estos sistemas interactivos,
caracterizados por la presencia de objetivos claros y un conjunto de elementos que
guían al usuario hacia su consecución, han demostrado ser herramientas poderosas
para captar la atención y fomentar la participación activa [2]. En este sentido, el juego se ha convertido en una actividad intrínsecamente atractiva para los jóvenes, tal como lo destacó Johan Huizinga en su obra [3], donde define el juego como “una actividad u ocupación voluntaria, ejercida dentro de ciertos límites de tiempo y espacio, que sigue reglas libremente aceptadas pero absolutamente obligatorias, y que va acompañada de un sentimiento de tensión y alegría, así como de una conciencia sobre su diferencia con la vida cotidiana”. 

La capacidad de los videojuegos para mantener la atención de los usuarios durante períodos de tiempo prolongados, junto con su potencial para desarrollar habilidades como el razonamiento lógico, la resolución de problemas y el trabajo colaborativo, los posiciona como recursos valiosos en el ámbito educativo. En este contexto, la gamificación emerge como una estrategia innovadora [4]. Según Marczewski (2015), en su obra Even Ninja Monkeys Like to Play, la gamificación se define como “el uso de ideas y elementos propios de los juegos en contextos ajenos a ellos, como el trabajo o la vida cotidiana”[5]. Este enfoque ha sido ampliamente adoptado en la educación, donde se ha demostrado su eficacia para incrementar la motivación intrínseca de los estudiantes y mejorar su compromiso con el proceso de aprendizaje.


La literatura especializada identifica los elementos de gamificación más recurrentes y efectivos: puntos, medallas y tablas de clasificación (leader-boards). Estudios como los de [6] y [7] respaldan la eficacia de estas mecánicas, destacando su capacidad para motivar cambios conductuales positivos en los estudiantes, quienes se ven incentivados a competir y superar desafíos para obtener recompensas. Es importante destacar que la gamificación no se limita al uso de videojuegos como tal, sino que se enfoca en aprovechar los elementos y dinámicas que estos incorporan. En el marco de esta investigación, para el desarrollo del proyecto, se tomará como base su teoría, los elementos que la componen y el valor que otorga a la motivación de los estudiantes.

La efectividad de la gamificación en educación no depende únicamente de su diseño
lúdico, sino de su capacidad para integrarse con principios pedagógicosfundamentales. En este sentido, teorías como el constructivismo, el aprendizaje basado en problemas y la teoría del flujo ofrecen un sustento teórico clave:
El constructivismo enfatiza la importancia de actividades que promuevan la
exploración, la experimentación y el aprendizaje autónomo, facilitando así la
construcción de conocimiento de forma personalizada y significativa [8]. 
El aprendizaje basado en problemas (ABP) enfoque fomenta un aprendizaje profundo y
significativo al vincular la teoría con la práctica de manera
interactiva [9].
La teoría del flujo explica cómo lograr inmersión y compromiso en actividades
educativas. El estado de flujo ocurre cuando hay equilibrio entre el desafío y las
habilidades del individuo, evitando aburrimiento (desafío bajo) o ansiedad (desafío
alto). Este balance es clave para diseñar experiencias educativas motivadoras y efectivas [10].

En el ámbito específico de la enseñanza de la programación, se han desarrollado
diversas herramientas y videojuegos educativos que buscan facilitar el aprendizaje de conceptos complejos. Entre los más destacados se encuentran:

\begin{itemize}
    \item Human Resource Machine (HRM): Es un videojuego educativo que enseña
programación básica mediante puzzles visuales donde el jugador escribe secuencias similares a código ensamblador, introduciendo conceptos básicos de la prgramación [11]; sin embargo, presenta limitaciones significativas al no ofrecer un seguimiento por parte de los docentes, se evalúa el cumplimiento de un contenido mediante la superación de niveles sin alguna validación extra, mantener una dificultad rígida sin adaptarse al ritmo individual y carecer de retroalimentación pedagógica ante errores. Estas limitaciones contradicen los principios del constructivismo al no fomentar una comprensión profunda mediante reflexión guiada, del aprendizaje basado en problemas al no promover estrategias de resolución autónoma con apoyo y de la teoría del flujo al no equilibrar desafío-habilidad de forma dinámica.

    \item Scratch: Es una plataforma de programación visual del MIT, introduce a los principiantes en el pensamiento computacional mediante proyectos creativos,
pero muestra importantes limitaciones: su diseño no incorpora mecanismos para el monitoreo docente ni evaluaciones que validen la comprensión conceptual más allá de la ejecución técnica, carece de progresión adaptativa que ajuste los desafíos al desarrollo individual de competencias, y ofrece retroalimentación básica centrada en errores sintácticos más que en el fortalecimiento de habilidades lógicas [12]. Estas deficiencias evidencian un desalineamiento tanto con la teoría del flujo y del aprendizaje basado en problemas al depender de una construcción externa de la narrativa o de los niveles, en vez de utilizar un sistema nativo.

\item Code Combat: Se presenta como un videojuego educativo que enseña programación mediante un enfoque de RPG, donde los estudiantes escriben
código real ya sea en JavaScript o Python para superar desafíos de fantasía
[11]. Pero el sistema no integra herramientas para que los docentes monitoreen
el progreso de sus estudiantes, y mantiene una progresión fija de la dificultad
que no se adapta al ritmo individual del aprendizaje. Estas carencias lo distancian del aprendizaje basado en problemas al estructurar los retos como
ejercicios cerrados más que como situaciones problemáticas auténticas y de la
teoría del flujo al no regular dinámicamente el equilibrio desafío-habilidad, lo
que reduce su efectividad como recurso educativo formal, particularmente en
contextos donde se requiere una evaluación formativa rigurosa y una
adaptación precisa a las necesidades de cada estudiante.

\item CodinGame: Es una plataforma interactiva que combina desafíos de
programación con mecánicas de juego, permitiendo a usuarios practicar con
múltiples lenguajes en contextos lúdicos [11]. Igual que en los videojuegos
anteriores carece de integración con sistemas de seguimiento docente y
mantiene retos estandarizados sin adaptación al nivel del usuario. Estas
deficiencias lo distancian de los principios constructivistas al no fomentar
construcción activa de conocimiento, del ABP al plantear ejercicios
desconectados de contextos reales y de la teoría del flujo al no regular la
progresión de dificultad, limitando su eficacia como herramienta educativa.
\end{itemize}

A partir de lo enunciado anteriormente, se identifica la siguiente situación
problemática:
Las soluciones existentes para el aprendizaje de programación, si bien son soluciones útiles, presentan limitaciones significativas. Entre estas, destaca la falta de un sistema que evalúe de manera efectiva el cumplimiento de los objetivos de aprendizaje en cada nivel, así como la ausencia de una retroalimentación clara y constructiva que guíe a los estudiantes en su proceso. Además, estas herramientas no proporcionan a los docentes un control adecuado para monitorear y apoyar el progreso de sus alumnos. Por otro lado, se observa una carencia en la adaptación de los niveles de dificultad, lo que impide que los estudiantes avancen de manera escalonada y acorde a su ritmo de aprendizaje. Estas deficiencias generan un vacío en la experiencia educativa, limitando tanto el desarrollo óptimo de los estudiantes como la capacidad de los profesores para intervenir de manera efectiva en su formación.

Las variables independientes sujetas a este proyecto, es el desarrollo del software con sus dos módulos para el aprendizaje de la programación de los estudiantes. La variable dependiente son los resultados del aprendizaje, si el estudio domina o no, los contenidos presentados en este proyecto.

Que se deriva en el presente \textbf{problema de investigación}:

¿Como implementar una plataforma educativa, que integre un aprendizaje mediante
videojuegos, el feedback sea inmediato y personalizado, además de que el nivel
pueda ajustarse al rendimiento del estudiante, así como garantizar el monitoreo de los resultados de los estudiantes por parte del profesor?

A raíz de este problema se propone el siguiente \textbf{objetivo general}:
-------

Para su cumplimiento se desglosa en los siguientes \textbf{objetivos específicos} y \textbf{tareas de investigación}:

\begin{enumerate}
    \item Objetivo 1
    \begin{enumerate}
        \item Tarea 1
    \end{enumerate}

    \item Objetivo 2
    \begin{enumerate}
        \item Tarea 2
    \end{enumerate}
    
    \item Objetivo 3
    \begin{enumerate}
        \item Tarea 3
    \end{enumerate}

\end{enumerate}


El \textbf{Valor Agregado} de este proyecto radica en

El presente documento está estructurado en 3 capitulos:

\begin{enumerate}
    \item Capitulo 1. \chapterOne
    \item Capitulo 2. \chapterSecond
    \item Capitulo 3. \chapterThree
\end{enumerate}

Según la revisión de la literatura realizada por [12], donde resume los principales tipo de artefactos IT, una descripción de un sistema, utilizando UML y que se enfoca en los aspectos de la estructura, los procesos y las interacciones, entrá en el espectro de Artefacto IT Diseño del Sistema. Siguiendo este ánalisis este proyecto de grado se clasifica como Artefacto IT Diseño del Sistema.





