\chapter{\chapterOne}
%\addcontentsline{toc}{chapter}{\chapterOne}
\chapter{Marco Teórico y Estado del Arte}
%\addcontentsline{toc}{chapter}{Marco Teórico y Estado del Arte}

El presente capítulo tiene como propósito establecer los fundamentos conceptuales y teóricos que sustentan el desarrollo de la plataforma educativa propuesta. A partir de una revisión de la literatura especializada, se abordan los principales enfoques que integran los videojuegos y la gamificación en los procesos de enseñanza-aprendizaje, así como las teorías psicológicas y pedagógicas que explican su efectividad.

Asimismo, se examinan los principios del aprendizaje adaptativo y la personalización educativa, esenciales para comprender el carácter dinámico y progresivo del sistema desarrollado. Finalmente, se exponen las bases teóricas que enmarcan el diseño arquitectónico del software educativo, permitiendo articular la dimensión tecnológica con la pedagógica del proyecto. Este marco teórico proporciona, por tanto, el sustento académico necesario para el diseño, implementación y validación de la solución planteada.

\section{Videojuegos y Aprendizaje}

La relación entre los videojuegos y el ámbito educativo ha sido descrita como compleja, marcada por etapas de aceptación, rechazo y evolución \cite{Gallego2002Panoramica}. Sin embargo, es innegable que los videojuegos constituyen una realidad ineludible en la sociedad contemporánea y representan una de las industrias más influyentes a nivel global \cite{Gallego2002Panoramica}. A lo largo de la historia, los juegos han desempeñado un papel fundamental en los procesos de aprendizaje \cite{Gallego2002Panoramica}, y su aplicación con fines educativos se ha consolidado como un campo de creciente interés en los últimos años \cite{EntrenaIngeniero}.

\subsection{Definición y características de los videojuegos}

Para comprender el papel educativo de los videojuegos, resulta necesario definir sus características esenciales y distinguir los tipos de juegos que se emplean en contextos de aprendizaje.

Un \textbf{juego} puede definirse como una prueba física o mental que se realiza conforme a reglas específicas, con el propósito de divertir, entrenar o recompensar al participante \cite{EntrenaIngeniero}. Desde una perspectiva más amplia, se entiende como un sistema en el cual los jugadores enfrentan un desafío abstracto definido por reglas, interactividad y retroalimentación, lo que produce un resultado cuantificable y, con frecuencia, una \textbf{reacción emocional} \cite{SanchezPacheco2020Enfoque}.  
En este sentido, un \textbf{videojuego} se concibe como una prueba mental ejecutada en un entorno computacional bajo ciertas reglas, cuyo fin es la diversión, el entretenimiento o la obtención de una recompensa \cite{EntrenaIngeniero}.

El videojuego puede considerarse también un \textbf{artefacto semiótico} o un \textbf{texto} en sí mismo \cite{LpezCanicio2019Cuando}. Un texto se define como un constructo comunicativo compuesto por un sistema de signos articulado en tres dimensiones: sintáctica, semántica y pragmática \cite{LpezCanicio2019Cuando}.  
Debido a su naturaleza interactiva, los videojuegos pueden entenderse como un \textbf{texto interactivo en lenguaje audiovisual (TILA)} \cite{LpezCanicio2019Cuando}.

\subsubsection{La interactividad como elemento central}

La característica más distintiva de los videojuegos, y la que los diferencia de otros medios narrativos tradicionales como la literatura o el cine, es la \textbf{interactividad} \cite{LpezCanicio2019Cuando}. Este componente constituye la esencia del medio videolúdico y le confiere su naturaleza participativa \cite{LpezCanicio2019Cuando}.

La interactividad permite que el productor diseñe o configure una narración que no solo es recibida e interpretada por el jugador, sino también \textbf{co-construida activamente} por él a través de sus decisiones y acciones \cite{LpezCanicio2019Cuando}.  
Este intercambio comunicativo bidireccional entre el texto audiovisual interactivo y el receptor transforma al jugador en un agente con capacidad de intervención, fenómeno conocido como \textbf{agencia} \cite{LpezCanicio2019Cuando}.  
En el plano narrativo, esta característica convierte al jugador en un \textbf{coautor} de la experiencia, desplazando su rol tradicional de receptor pasivo hacia una posición creativa y participativa \cite{LpezCanicio2019Cuando}.

\subsubsection{Tipologías educativas}

En el ámbito educativo, la tipología de videojuegos más destacada es el \textbf{videojuego serio} o \textit{serious game} \cite{EntrenaIngeniero}.  
Estos juegos se distinguen por emplear la diversión como medio de formación, con objetivos específicos en áreas como la educación, la salud o la comunicación estratégica \cite{EntrenaIngeniero}.  
La relevancia educativa de los \textit{serious games} radica en que su propósito trasciende el entretenimiento, integrando el aprendizaje significativo como objetivo central \cite{EntrenaIngeniero}.  
En el contexto universitario, este tipo de juegos digitales ha adquirido un papel protagónico como recurso pedagógico \cite{SierraDaza2024Videojuegos}, ya que permiten simular escenarios complejos y realistas que favorecen la comprensión profunda de procesos y la resolución de problemas contextualizados \cite{SierraDaza2024Videojuegos}.

\subsection{Aprendizaje Basado en Juegos (ABJ)}

El \textbf{Aprendizaje Basado en Juegos (ABJ)} o \textit{Game-Based Learning (GBL)} se define como la utilización, creación o adaptación de juegos —incluidos los videojuegos y aplicaciones con fines educativos— en el contexto del aula \cite{SierraDaza2024Videojuegos}.  
Este enfoque se reconoce como un recurso que facilita el aprendizaje activo y significativo \cite{SierraDaza2024Videojuegos}.

En la educación universitaria, el ABJ destaca por su capacidad para \textbf{fomentar el compromiso y la participación activa del estudiante} \cite{SierraDaza2024Videojuegos}. Numerosos estudios señalan una correlación positiva entre las actividades lúdicas y la adquisición de conocimientos \cite{SierraDaza2024Videojuegos}.  
La efectividad del ABJ se sustenta en la integración de cinco elementos esenciales \cite{SierraDaza2024Videojuegos}:

\begin{enumerate}
    \item \textbf{Motivación:} el juego estimula la disposición y el interés por aprender.  
    \item \textbf{Aprendizaje divertido:} el disfrute actúa como catalizador del aprendizaje.  
    \item \textbf{Autonomía:} promueve la exploración y la toma de decisiones independientes.  
    \item \textbf{Autenticidad:} propicia la conexión entre la experiencia lúdica y el aprendizaje significativo.  
    \item \textbf{Aprendizaje experiencial:} el estudiante aprende \textbf{haciendo a través del juego}.
\end{enumerate}

Es importante diferenciar el ABJ de la \textbf{gamificación} o \textit{ludificación}, entendida como la aplicación de elementos y dinámicas del diseño de juegos en contextos no lúdicos, como el educativo, con el objetivo de potenciar la motivación, el compromiso y la participación del estudiante \cite{OrtizColon2018Gamificacion, Gallego2002Panoramica, BeltranMorales2017Elearning}.

\subsection{Beneficios y desafíos del uso de videojuegos en educación}

El empleo de videojuegos con fines educativos ha sido ampliamente estudiado, revelando beneficios notables para el proceso de enseñanza-aprendizaje, aunque también presenta desafíos pedagógicos, técnicos y logísticos.

\subsubsection{Beneficios educativos}

Diversas investigaciones coinciden en que los videojuegos incrementan la satisfacción, la motivación y la retención del conocimiento \cite{Gallego2002Panoramica, EntrenaIngeniero, PadronGonzalez2023ElUso}.  
Su carácter interactivo convierte el aprendizaje en una experiencia activa, inmersiva y enriquecedora \cite{PadronGonzalez2023ElUso}.  
Entre los principales beneficios destacan:

\begin{itemize}
    \item \textbf{Aumento de la motivación y el compromiso:} los videojuegos captan la atención del estudiante y fomentan la participación sostenida \cite{Gallego2002Panoramica, PadronGonzalez2023ElUso, SierraDaza2024Videojuegos}. La motivación, motor esencial del aprendizaje, se ve reforzada por la sensación de progreso y logro.
    \item \textbf{Aprendizaje activo y experiencial:} los estudiantes aprenden experimentando, probando y reflexionando sobre sus acciones, lo que favorece la asimilación de conceptos y el desarrollo de habilidades cognitivas \cite{Gallego2002Panoramica, SmithBasak2023Meta}.  
    \item \textbf{Retroalimentación inmediata y reducción del miedo al error:} los videojuegos permiten equivocarse sin consecuencias negativas y ofrecen una respuesta instantánea tras cada acción, promoviendo el aprendizaje por ensayo y error \cite{Gallego2002Panoramica, TheRoleofPerceivedRelevance2018}.
    \item \textbf{Desarrollo de habilidades transversales:} potencian competencias como la resolución de problemas, la toma de decisiones, la creatividad y el trabajo colaborativo \cite{SierraDaza2024Videojuegos, PadronGonzalez2023ElUso}.  
    \item \textbf{Adquisición de conocimiento específico:} en el contexto universitario, la mayoría de los estudios reportan que los videojuegos contribuyen significativamente a la adquisición y comprensión de contenidos disciplinares \cite{SierraDaza2024Videojuegos}.  
    \item \textbf{Inmersión y autonomía:} generan una implicación total del estudiante en la actividad y fortalecen su sentido de control sobre el proceso de aprendizaje \cite{Gallego2002Panoramica, EntrenaIngeniero}.
\end{itemize}

\subsubsection{Desafíos y problemáticas}

A pesar de los beneficios señalados, la integración de videojuegos en los entornos educativos implica superar diversos desafíos \cite{PadronGonzalez2023ElUso}:

\begin{itemize}
    \item \textbf{Uso excesivo y distracción:} es necesario regular el tiempo de exposición y evitar la dependencia o la pérdida de concentración en los objetivos educativos \cite{PadronGonzalez2023ElUso}.  
    \item \textbf{Equilibrio y supervisión docente:} los videojuegos deben emplearse como complemento pedagógico, bajo una planificación y guía adecuadas \cite{PadronGonzalez2023ElUso}.  
    \item \textbf{Inversión y diseño específico:} el desarrollo de videojuegos educativos requiere recursos, tiempo y un diseño pedagógico coherente con los objetivos formativos \cite{Gallego2002Panoramica, EntrenaIngeniero}.  
    \item \textbf{Infraestructura y acceso:} las limitaciones tecnológicas y de conectividad pueden restringir su implementación en algunos contextos educativos \cite{PadronGonzalez2023ElUso}.  
    \item \textbf{Formación docente y resistencia al cambio:} la integración efectiva requiere capacitación del profesorado y superación de prejuicios hacia los videojuegos como herramientas de aprendizaje \cite{PadronGonzalez2023ElUso}.  
    \item \textbf{Evaluación del desempeño:} medir el impacto educativo de los videojuegos sigue siendo un desafío, especialmente cuando no fueron diseñados con fines pedagógicos específicos \cite{PadronGonzalez2023ElUso}.
\end{itemize}

Para aprovechar plenamente su potencial, se requiere una metodología teórico-práctica cuidadosamente planificada, que equilibre la dimensión lúdica y la pedagógica, garantizando así su eficacia en los procesos de enseñanza-aprendizaje \cite{PadronGonzalez2023ElUso}.

\noindent
Tras comprender el papel de los videojuegos como medio cultural, narrativo y pedagógico, es necesario profundizar en una de las estrategias que ha tomado mayor relevancia en la última década: la \textbf{gamificación}.  
Si bien los videojuegos educativos y el Aprendizaje Basado en Juegos (ABJ) se centran en el uso directo de juegos para promover la adquisición de conocimientos, la gamificación trasciende este enfoque al incorporar las dinámicas, mecánicas y elementos propios del juego en entornos que no son lúdicos por naturaleza.  

En otras palabras, mientras los videojuegos buscan enseñar \textit{a través del juego}, la gamificación busca enseñar \textit{como si fuera un juego}.  
Esta distinción marca un cambio de paradigma dentro de la educación moderna, al pasar de la simple utilización del juego como herramienta didáctica a la integración de sus principios estructurales en el diseño de experiencias de aprendizaje.  

A continuación, se explora el concepto de gamificación desde una perspectiva teórica y pedagógica, abordando sus definiciones, elementos constitutivos, teorías de respaldo y evidencia empírica de su efectividad en contextos educativos.

% ---------------------------------------------

\section{Gamificación}

La \textbf{gamificación} ha emergido en los últimos años como una de las estrategias más prometedoras dentro de la innovación educativa, impulsada por el desarrollo de las Tecnologías de la Información y la Comunicación (TIC) y por la necesidad de adaptar los procesos de enseñanza-aprendizaje a las nuevas generaciones digitales \cite{OrtizColon2018Gamificacion, Gallego2002Panoramica, Impact of Gamification on Motivation and Academic, BeltranMorales2017Elearning}.  
Esta metodología busca integrar en entornos no lúdicos los principios, dinámicas y elementos propios de los juegos, con el propósito de potenciar la motivación, el compromiso y el rendimiento de los estudiantes.

\subsection{Definición y diferencias con el Aprendizaje Basado en Juegos}

El término \textbf{gamificación} (también denominado \textit{ludificación}) se define como el \textit{uso de mecánicas y dinámicas de juego en contextos no lúdicos} \cite{BeltranMorales2017Elearning, OrtizColon2018Gamificacion}, con el fin de promover la motivación, la concentración, el esfuerzo y la fidelización \cite{BeltranMorales2017Elearning}.  
En esencia, consiste en trasladar la lógica de los juegos a ámbitos como la educación, la empresa o la salud, aprovechando la predisposición natural de las personas a participar, competir y superar retos \cite{Gallego2002Panoramica}.

Desde una perspectiva más amplia, la gamificación puede entenderse como la \textbf{aplicación del pensamiento del jugador y de técnicas de diseño de juegos} para atraer a los usuarios, fomentar su participación y resolver problemas de manera creativa \cite{BeltranMorales2017Elearning, A_Systematic_Review_of_Gamification_Research}.  
Su finalidad no es convertir el aprendizaje en un juego, sino incorporar sus principios psicológicos —progreso, recompensa, autonomía y reto— para fortalecer el proceso educativo.

\subsubsection{Diferencias con el Aprendizaje Basado en Juegos (ABJ)}

Es importante distinguir la gamificación del \textbf{Aprendizaje Basado en Juegos (ABJ)} y de los videojuegos educativos.  
Mientras el ABJ emplea juegos completos como herramienta de enseñanza, la gamificación extrae solo ciertos elementos del diseño de juegos para integrarlos en un contexto no lúdico \cite{SierraDaza2024Videojuegos, BeltranMorales2017Elearning, OrtizColon2018Gamificacion}.  

Las diferencias principales se resumen así:

\begin{enumerate}
    \item \textbf{Contexto de aplicación:} El ABJ utiliza juegos (digitales o analógicos) como medio de aprendizaje, mientras que la gamificación incorpora mecánicas de juego en actividades cotidianas o académicas \cite{BeltranMorales2017Elearning}.  
    \item \textbf{Objetivo principal:} En los videojuegos el fin es el entretenimiento; en la gamificación, el objetivo es \textbf{modificar actitudes y fomentar la motivación y el compromiso} \cite{BeltranMorales2017Elearning}.  
    \item \textbf{Integración curricular:} Los juegos educativos buscan enseñar contenidos específicos; la gamificación, en cambio, puede aplicarse de forma transversal para fortalecer la implicación del estudiante y su aprendizaje autónomo \cite{BeltranMorales2017Elearning}.
\end{enumerate}

\subsection{Elementos comunes de la gamificación: PBL, progresión y narrativa}

Los fundamentos de la gamificación se estructuran en tres niveles jerárquicos —\textbf{Dinámicas}, \textbf{Mecánicas} y \textbf{Componentes}— propuestos por Werbach y Hunter (2012) \cite{OrtizColon2018Gamificacion, BeltranMorales2017Elearning}.  
Estos niveles permiten comprender cómo se construyen las experiencias gamificadas:

\begin{table}[H]
\centering
\begin{tabular}{|p{3cm}|p{7cm}|p{4cm}|}
\hline
\textbf{Categoría} & \textbf{Descripción} & \textbf{Ejemplos en educación} \\ \hline
\textbf{Dinámicas} & Aspectos abstractos que responden a deseos humanos: logro, pertenencia o autonomía. & Narrativa, progresión, emociones, interacción social, restricciones \cite{OrtizColon2018Gamificacion, BeltranMorales2017Elearning}. \\ \hline
\textbf{Mecánicas} & Procesos que impulsan la acción y estructuran la experiencia del jugador. & Retos, recompensas, competencia, cooperación, retroalimentación \cite{OrtizColon2018Gamificacion}. \\ \hline
\textbf{Componentes} & Implementaciones visibles de las mecánicas y dinámicas. & Puntos, niveles, insignias, rankings, avatares, barras de progreso \cite{OrtizColon2018Gamificacion, BeltranMorales2017Elearning}. \\ \hline
\end{tabular}
\caption{Categorías principales de la gamificación. Fuente: Adaptado de Werbach y Hunter (2012).}
\end{table}

Entre los elementos más comunes, los \textbf{niveles} y \textbf{puntos} son los más empleados, seguidos por las \textbf{insignias} y las \textbf{tablas de clasificación} \cite{2106.09942v1}.  
Otros componentes clave incluyen la \textbf{retroalimentación}, las \textbf{metas}, la \textbf{narrativa} y las \textbf{barras de progreso}, los cuales favorecen el sentido de avance y pertenencia dentro de la experiencia gamificada.

\subsubsection{Progresión y narrativa}

La \textbf{progresión} constituye una dinámica esencial, pues permite mantener el interés del estudiante al ofrecer un sentido claro de avance y superación \cite{BeltranMorales2017Elearning}.  
Los \textbf{niveles} simbolizan el crecimiento del jugador, mientras que la \textbf{narrativa} proporciona el contexto que convierte el aprendizaje en una experiencia significativa, otorgando al estudiante el rol de protagonista dentro de una historia \cite{OrtizColon2018Gamificacion, BeltranMorales2017Elearning}.

\subsection{La motivación como eje de la gamificación}

La motivación constituye el núcleo conceptual de la gamificación \cite{OrtizColon2018Gamificacion, A_Systematic_Review_of_Gamification_Research, Impact of Gamification on Motivation and Academic}.  
Motivar implica despertar el interés y la energía interna del estudiante hacia una actividad, manteniendo su compromiso y esfuerzo sostenido \cite{OrtizColon2018Gamificacion}.  

La literatura distingue dos tipos principales de motivación \cite{OrtizColon2018Gamificacion}:

\begin{itemize}
    \item \textbf{Motivación extrínseca:} Proviene de recompensas externas, como calificaciones, insignias o reconocimientos \cite{BeltranMorales2017Elearning}.  
    \item \textbf{Motivación intrínseca:} Surge del interés personal y del placer de realizar la actividad por sí misma \cite{OrtizColon2018Gamificacion}.  
\end{itemize}

El objetivo de una experiencia gamificada eficaz es \textbf{transformar la motivación extrínseca en intrínseca}, logrando que el estudiante aprenda por satisfacción personal y no solo por recompensa \cite{SanchezPacheco2020Enfoque}.  
La \textbf{autonomía} en la toma de decisiones y la percepción de competencia son factores clave para este cambio \cite{Gallego2002Panoramica}.

Entre las teorías que sustentan el estudio de la motivación destacan:

\begin{itemize}
    \item \textbf{Teoría de la Autodeterminación (Self-Determination Theory, SDT):} Explica cómo la gamificación puede potenciar la motivación intrínseca al satisfacer las necesidades psicológicas básicas de autonomía, competencia y relación social \cite{A_Systematic_Review_of_Gamification_Research, Impact of Gamification on Motivation and Academic}.  
    \item \textbf{Teoría del Flujo (Flow Theory):} Describe el estado de concentración total e inmersión que experimenta una persona al realizar una tarea desafiante y significativa. Una gamificación bien diseñada busca inducir este estado para mejorar el aprendizaje y el rendimiento \cite{OrtizColon2018Gamificacion}.  
\end{itemize}

\subsection{Evidencia de efectividad en contextos educativos}

La investigación reciente confirma la eficacia de la gamificación en el ámbito educativo. Se estima que el 45,19\% de los estudios revisados se centran en aplicaciones de gamificación en educación \cite{A_Systematic_Review_of_Gamification_Research, Impact of Gamification on Motivation and Academic}.  

\subsubsection{Beneficios comprobados}

Los resultados más recurrentes señalan \textbf{incrementos significativos en la motivación, el compromiso y el rendimiento académico} \cite{Impact of Gamification on Motivation and Academic, OrtizColon2018Gamificacion, SierraDaza2024Videojuegos}.  
También se destaca el desarrollo de habilidades cognitivas y sociales, la reducción del miedo al error gracias a la retroalimentación inmediata, y el fortalecimiento de la autonomía del estudiante \cite{BeltranMorales2017Elearning, Gallego2002Panoramica}.

\subsubsection{Desafíos y consideraciones}

Pese a sus beneficios, la gamificación presenta riesgos si se aplica de manera superficial. Un diseño deficiente puede reducir su impacto o generar dependencia de recompensas extrínsecas \cite{SanchezPacheco2020Enfoque}.  
Por ello, se recomienda que la experiencia gamificada mantenga un equilibrio entre \textbf{recompensa, reto y autonomía}, adaptando la dificultad de las tareas al nivel de competencia del estudiante \cite{BeltranMorales2017Elearning}.

\subsection{Categorías y marcos de diseño de la gamificación}

El diseño de experiencias gamificadas requiere planificación estratégica y coherencia pedagógica \cite{OrtizColon2018Gamificacion}.  
Entre los marcos teóricos más relevantes destacan:

\begin{itemize}
    \item \textbf{Framework MDA (Mecánicas, Dinámicas y Estéticas):} Permite analizar cómo los diferentes elementos de diseño contribuyen a las respuestas emocionales y cognitivas del jugador \cite{A_Systematic_Review_of_Gamification_Research}.  
    \item \textbf{Modelo D6 (Werbach):} Propone seis pasos para diseñar estrategias gamificadas: definir objetivos, identificar conductas deseadas, perfilar jugadores, estructurar bucles de compromiso, incorporar diversión y seleccionar las herramientas adecuadas \cite{BeltranMorales2017Elearning}.  
\end{itemize}

Finalmente, algunos autores proponen considerar la gamificación como una \textbf{nueva teoría del aprendizaje}, ya que integra principios motivacionales, cognitivos y conductuales que complementan las teorías tradicionales del aprendizaje \cite{SanchezPacheco2020Enfoque}.


% ----------------------------------------------

\section{Aprendizaje Adaptativo y Personalización}
\subsection{Definición de aprendizaje adaptativo}
\subsection{Mecanismos de personalización en entornos digitales}
\subsection{Relación entre adaptación, motivación y rendimiento académico}

% ---------------------------------------------

\section{Ejemplos de Videojuegos Educativos Destacados}

% ---------------------------------------------

\section{Videojuegos Educativos en Cuba}

% ---------------------------------------------

\section{Desafíos y Riesgos del Uso de Videojuegos en el Aprendizaje}

% ---------------------------------------------


\section{Teorías del Aprendizaje}

\subsection{Constructivismo}

\subsection{Aprendizaje basado en problemas}

% ---------------------------------------------

\section{Estado de Flujo}
\subsection{Características del estado de flujo}
\subsection{Condiciones para alcanzar el flujo en entornos educativos}
\subsection{Relación entre flujo, gamificación y engagement}

% ---------------------------------------------

\section{Selección de Tecnologías utilizadas}

\subsection{Base de Datos}

\subsection{Módulo del Estudiante}

\subsection{Módulo del Profesor}

% ---------------------------------------------

\section{Conclusiones parciales}
