\chapter{\chapterOne}
%\addcontentsline{toc}{chapter}{\chapterOne}
\chapter{Marco Teórico y Estado del Arte}
%\addcontentsline{toc}{chapter}{Marco Teórico y Estado del Arte}

El presente capítulo tiene como propósito establecer los fundamentos conceptuales y teóricos que sustentan el desarrollo de la plataforma educativa propuesta. A partir de una revisión de la literatura especializada, se abordan los principales enfoques que integran los videojuegos y la gamificación en los procesos de enseñanza-aprendizaje, así como las teorías psicológicas y pedagógicas que explican su efectividad.

Asimismo, se examinan los principios del aprendizaje adaptativo y la personalización educativa, esenciales para comprender el carácter dinámico y progresivo del sistema desarrollado. Finalmente, se exponen las bases teóricas que enmarcan el diseño arquitectónico del software educativo, permitiendo articular la dimensión tecnológica con la pedagógica del proyecto. Este marco teórico proporciona, por tanto, el sustento académico necesario para el diseño, implementación y validación de la solución planteada.

\section{Videojuegos y Aprendizaje}

\subsection{Definición y características de los videojuegos}
\subsection{Aprendizaje Basado en Juegos}
\subsection{Beneficios y desafíos del uso de videojuegos en educación}

% ---------------------------------------------

\section{Gamificación}
\subsection{Definición y diferencias con Game-Based Learning}
\subsection{Elementos comunes de la gamificación (PBL, progresión, narrativa)}
\subsection{La Motivación}
\subsection{Evidencia de efectividad en contextos educativos}
\subsection{Categorías de la Gamificación}

% ----------------------------------------------
\section{Aprendizaje Adaptativo y Personalización}
\subsection{Definición de aprendizaje adaptativo}
\subsection{Mecanismos de personalización en entornos digitales}
\subsection{Relación entre adaptación, motivación y rendimiento académico}

% ---------------------------------------------

\section{Ejemplos de Videojuegos Educativos Destacados}
\section{Videojuegos Educativos en Cuba}

% ---------------------------------------------

\section{Desafíos y Riesgos del Uso de Videojuegos en el Aprendizaje}

% ---------------------------------------------


\section{Teorías del Aprendizaje}

\subsection{Constructivismo}

\subsection{Aprendizaje basado en problemas}

% ---------------------------------------------

\section{Estado de Flujo}
\subsection{Características del estado de flujo}
\subsection{Condiciones para alcanzar el flujo en entornos educativos}
\subsection{Relación entre flujo, gamificación y engagement}

% ---------------------------------------------

\section{Selección de Tecnologías utilizadas}

\subsection{Base de Datos}

\subsection{Módulo del Estudiante}

\subsection{Módulo del Profesor}

% ---------------------------------------------

\section{Conclusiones parciales}
