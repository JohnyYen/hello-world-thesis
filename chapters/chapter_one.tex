\chapter{\chapterOne}
%\addcontentsline{toc}{chapter}{\chapterOne}
\chapter{Marco Teórico y Estado del Arte}
%\addcontentsline{toc}{chapter}{Marco Teórico y Estado del Arte}

\section{Videojuegos y Aprendizaje}

\subsection{Definición y características de los videojuegos}
\subsection{Aprendizaje Basado en Juegos}
\subsection{Beneficios y desafíos del uso de videojuegos en educación}
Esto es una prueba de cita \cite{example_book}.

% ---------------------------------------------

\section{Gamificación}
\subsection{Definición y diferencias con Game-Based Learning}
\subsection{Elementos comunes de la gamificación (PBL, progresión, narrativa)}
\subsection{La Motivación}
\subsection{Evidencia de efectividad en contextos educativos}

\subsection{Categorías de la Gamificación}

% ---------------------------------------------

\section{Ejemplos de Videojuegos Educativos Destacados}
\section{Videojuegos Educativos en Cuba}

% ---------------------------------------------

\section{Desafíos y Riesgos del Uso de Videojuegos en el Aprendizaje}

% ---------------------------------------------


\section{Teorías del Aprendizaje}

\subsection{Constructivismo}

\subsection{Aprendizaje basado en problemas}

% ---------------------------------------------

\section{Estado de Flujo}
\subsection{Características del estado de flujo}
\subsection{Condiciones para alcanzar el flujo en entornos educativos}
\subsection{Relación entre flujo, gamificación y engagement}

% ---------------------------------------------

\section{Tecnologías utilizadas}

\subsection{Base de Datos}

\subsection{Módulo del Estudiante}

\subsection{Módulo del Profesor}

% ---------------------------------------------

\section{Conclusiones parciales}
