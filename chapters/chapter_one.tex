\chapter{\chapterOne}
%\addcontentsline{toc}{chapter}{\chapterOne}
<<<<<<< HEAD

hola \cite{knuth1990}
=======
\chapter*{Introducción}
\addcontentsline{toc}{chapter}{Introducción}

En el contexto del siglo XXI, uno de los desafíos más persistentes en el ámbito educativo es la necesidad de adaptar las metodologías de enseñanza a las demandas de una sociedad en constante evolución. A pesar de los avances tecnológicos y pedagógicos, la educación enfrenta retos significativos en todos los niveles escolares, particularmente en la enseñanza universitaria.

Los videojuegos, como fenómeno cultural y social, han adquirido un papel  central en la vida de millones de jóvenes a nivel global. Estos sistemas interactivos, caracterizados por la presencia de objetivos claros y un conjunto de  elementos que guían al usuario hacia su consecución, han demostrado ser  herramientas poderosas para captar la atención y fomentar la participación  activa. Su capacidad para mantener la atención de los usuarios durante períodos de tiempos prolongados, junto con su potencial para desarrollar habilidades como razonamiento lógico, resolución de problemas, entre otros, los posiciona como recurso valioso en el ámbito educativo. En este sentido, el juego se ha convertido en una actividad intrínseca

La capacidad de los videojuegos para mantener la atención de los usuarios
durante períodos de tiempo prolongados, junto con su potencial para desarro-
llar habilidades como el razonamiento lógico, la resolución de problemas y el
trabajo colaborativo, los posiciona como recursos valiosos en el ámbito educa-
tivo. En este contexto, la gamificación emerge como una estrategia innovadora
[4]. Según Marczewski (2015), en su obra Even Ninja Monkeys Like to Play, la
gamificación se define como “el uso de ideas y elementos propios de los jue-
gos en contextos ajenos a ellos, como el trabajo o la vida cotidiana”[5]. Este
enfoque ha sido ampliamente adoptado en la educación, donde se ha demos-
trado su eficacia para incrementar la motivación intrínseca de los estudiantes
y mejorar su compromiso con el proceso de aprendizaje.

La literatura especializada identifica los elementos de gamificación más
recurrentes y efectivos: puntos, medallas y tablas de clasificación (leader-boards). Estudios como los de [6] y [7] respaldan la eficacia de estas mecá-
nicas, destacando su capacidad para motivar cambios conductuales positivos
en los estudiantes, quienes se ven incentivados a competir y superar desafíos
para obtener recompensas.

Es importante destacar que la gamificación no se limita al uso de video-
juegos como tal, sino que se enfoca en aprovechar los elementos y dinámicas
que estos incorporan. En el marco de esta investigación, para el desarrollo del
proyecto, se tomará como base su teoría, los elementos que la componen y el
valor que otorga a la motivación de los estudiantes.

La efectividad de la gamificación en educación no depende únicamente
de su diseño lúdico, sino de su capacidad para integrarse con principios pe-
dagógicos fundamentales. En este sentido, teorías como el constructivismo,
el aprendizaje basado en problemas y la teoría del flujo ofrecen un sustento
teórico clave
% Hablar de gamificación y su relación con los videojuegos

% Hablar de las leyes y reglas psicologicas del aprendizaje y su relación con los videojuegos

El constructivismo enfatiza la importancia de actividades que promuevan
la exploración, la experimentación y el aprendizaje autónomo, facilitando así
la construcción de conocimiento de forma personalizada y significativa [8].
El aprendizaje basado en problemas (ABP) enfoque fomenta un apren-
dizaje profundo y significativo al vincular la teoría con la práctica de manera
interactiva [9].

La teoría del flujo explica cómo lograr inmersión y compromiso en acti-
vidades educativas. El estado de flujo ocurre cuando hay equilibrio entre el
desafío y las habilidades del individuo, evitando aburrimiento (desafío bajo) o
ansiedad (desafío alto). Este balance es clave para diseñar experiencias edu-
cativas motivadoras y efectivas [10].

% Ejemplos de videojuegos interesantes

% - Human Resource Machine

A partir de lo enunciado anteriormente, se identifica el siguiente \textbf{problema de investigación}:

Las soluciones existentes para el aprendizaje de programación, si bien son soluciones útiles e interesantes, presentan limitaciones significativas. Entre estas, destaca la falta de un sistema que evalúe de manera efectiva el cumplimiento de los objetivos de aprendizaje en cada nivel, así como la ausencia de una retroalimentación clara y constructiva que guíe a los estudiantes en su proceso. Además, estas herramientas no proporcionan a los docentes un control adecuado para monitorear y apoyar el progreso de sus alumnos. Por otro lado, se observa una carencia en la adaptación de los niveles de dificultad, lo que impide que los estudiantes avancen de manera escalonada y acorde a su ritmo de aprendizaje. Estas deficiencias generan un vacío en la experiencia educativa, limitando tanto el desarrollo óptimo de los estudiantes como la capacidad de los profesores para intervenir de manera efectiva en su formación.

A partir de la problemática anteriormente planteada se propone el siguiente \textbf{objetivo general}:

% Objetivo general

% Objetivos específicos y tareas de investigacion

El \textbf{valor agregado} de este proyecto radica en el desarrollo de un videojuego educativo capaz de medir automáticamente el progreso de los estudiantes,  incluso cuando superan un nivel determinado. Esta herramienta no solo se  adaptará a las características individuales de cada estudiante, sino que también proporcionará retroalimentación oportuna y personalizada, fortaleciendo  así su proceso de aprendizaje. Además, contará con un módulo especializado  que permitirá a los profesores realizar un seguimiento detallado del desempeño de sus alumnos, facilitando la identificación de áreas de mejora y el diseño  de estrategias pedagógicas más efectivas.


El presente documento está estructurado en tres capítulos:

\begin{enumerate}ddf
	\item \textbf{Capitulo 1. Marco Teorico y Estado del Arte}
	
	\item \textbf{Capitulo 2. Diseño del Sistema}
	
	\item \textbf{Validación del sistema}
\end{enumerate}
>>>>>>> 4d45132 (setup introduction)
