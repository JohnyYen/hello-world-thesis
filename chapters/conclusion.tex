\chapter*{Conclusiones}
\addcontentsline{toc}{chapter}{Conclusiones}

A lo largo de esta investigación se abordaron los objetivos específicos planteados, y cada uno de ellos fue alcanzado mediante un proceso riguroso de análisis, diseño, implementación y validación. A continuación, se presentan las conclusiones generales organizadas según los objetivos específicos de la tesis.

\begin{itemize}
    \item \textbf{En relación con el análisis de los fundamentos teóricos y los antecedentes del uso de videojuegos educativos y plataformas de programación}, se concluye que la literatura respalda ampliamente el potencial de los videojuegos como herramientas pedagógicas para promover la motivación, la participación activa y el aprendizaje significativo. El estudio de teorías educativas como el constructivismo, el aprendizaje basado en problemas y la gamificación permitió fundamentar de manera sólida las decisiones de diseño del videojuego. De igual forma, la revisión de experiencias nacionales e internacionales evidenció la necesidad de propuestas que integren elementos narrativos, retroalimentación inmediata y adaptación de dificultad, aspectos que fueron incorporados en el prototipo desarrollado.

    \item \textbf{Respecto al diseño del videojuego educativo basado en principios de gamificación}, se logró definir un conjunto de mecánicas y dinámicas que favorecen la interacción significativa del estudiante con los contenidos. La selección del género del videojuego y la definición de su narrativa respondieron directamente a los objetivos educativos planteados, garantizando coherencia entre las actividades propuestas y los aprendizajes esperados. La narrativa, los elementos visuales y la estructuración de niveles permitieron contextualizar los problemas de programación en escenarios cotidianos, aumentando la inmersión y facilitando la comprensión del contenido. El diseño final resultó en un entorno atractivo, motivador y alineado con objetivos pedagógicos claros.

    \item \textbf{En cuanto al diseño de la arquitectura tecnológica que integra el videojuego y la plataforma web}, se estableció una estructura sólida compuesta por módulos independientes que interactúan entre sí de forma coherente y eficiente. Se definieron requerimientos funcionales y no funcionales que guiaron la construcción de la solución, asegurando criterios como escalabilidad, mantenibilidad y claridad en los flujos de información. El modelo de datos, los componentes del videojuego, los servicios de la plataforma y las rutas de comunicación quedaron articulados dentro de una arquitectura integral capaz de soportar la gestión de usuarios, el seguimiento del desempeño y la adaptación automatizada de los niveles del juego.

    \item \textbf{Finalmente, respecto a la implementación y evaluación del prototipo mediante pruebas técnicas y pedagógicas}, se completó el desarrollo funcional del videojuego educativo y se aplicaron pruebas exhaustivas para validar su comportamiento. Las pruebas funcionales, de rendimiento, unitarias y de integración demostraron la estabilidad del sistema, el correcto funcionamiento de sus módulos principales y la coherencia de los flujos educativos implementados. Asimismo, la aplicación de pruebas piloto y la recolección de retroalimentación de usuarios permitieron identificar fortalezas del prototipo, como la claridad de los niveles y la utilidad del sistema de retroalimentación, así como oportunidades de mejora que orientan futuras versiones del proyecto. El análisis general de resultados confirma que el prototipo es viable, efectivo y posee potencial para evolucionar hacia una herramienta educativa completa.
\end{itemize}

En síntesis, la investigación permitió demostrar que el uso de videojuegos educativos, respaldado por teoría pedagógica y sustentado en una arquitectura tecnológica robusta, constituye una estrategia viable para apoyar el aprendizaje de programación en entornos interactivos. El prototipo desarrollado integra narrativa, adaptación automatizada, retroalimentación significativa y ejecución basada en bloques, elementos que en conjunto favorecen una experiencia educativa motivadora y formativa. El cumplimiento de los objetivos planteados valida la pertinencia del proyecto y sienta las bases para futuras mejoras, ampliaciones y evaluaciones a mayor escala.

