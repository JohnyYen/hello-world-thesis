\chapter*{Resumen}

\textbf{Palabras Claves: } ...

\newpage

\chapter*{Abstract}


\textbf{Keywords: } ...

% ------------------------------
% Resumen en Español
% ------------------------------
% \begin{abstract}
% El presente documento propone el diseño de una \textbf{plataforma educativa interactiva} orientada al aprendizaje de la programación mediante un \textbf{videojuego educativo adaptativo} y un \textbf{sistema web de gestión académica}. El videojuego permite a los estudiantes practicar conceptos de programación en un entorno lúdico y personalizado, ajustando los niveles y desafíos según el rendimiento individual. El módulo web para profesores facilita la supervisión del progreso de los estudiantes, la generación de reportes y la toma de decisiones pedagógicas basadas en datos. El sistema se sustenta en teorías de aprendizaje basadas en juegos, gamificación, aprendizaje adaptativo y constructivismo, con el objetivo de mejorar la motivación, la retención de conocimientos y la experiencia educativa en general.

% \textbf{Palabras clave:} aprendizaje de programación, videojuego educativo, gamificación, aprendizaje adaptativo, plataforma educativa, gestión académica.
% \end{abstract}

% % ------------------------------
% % Abstract in English
% % ------------------------------
% % \begin{abstract}
% % This document presents the design of an \textbf{interactive educational platform} aimed at teaching programming through an \textbf{adaptive educational game} and a \textbf{web-based academic management system}. The game enables students to practice programming concepts in a playful and personalized environment, adjusting levels and challenges based on individual performance. The web module for teachers allows monitoring student progress, generating reports, and making data-driven pedagogical decisions. The system is grounded in game-based learning, gamification, adaptive learning, and constructivist theories, aiming to enhance motivation, knowledge retention, and the overall educational experience.

% % \textbf{Keywords:} programming learning, educational game, gamification, adaptive learning, educational platform, academic management.
% % \end{abstract}

