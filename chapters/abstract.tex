\chapter*{Resumen}

La presente investigación propone el diseño e implementación de una plataforma educativa basada en videojuegos adaptativos para la enseñanza de la programación. Partiendo de las limitaciones de las metodologías tradicionales, se plantea una arquitectura de software que combina los principios de la gamificación, el aprendizaje constructivista y el aprendizaje adaptativo, con el fin de incrementar la motivación y la eficacia en el proceso formativo. La solución está compuesta por un videojuego educativo y un entorno web que permite a los docentes monitorear el progreso de los estudiantes. El videojuego, desarrollado bajo el motor Godot 4, integra mecánicas de resolución de problemas mediante bloques de código visuales, contextualizados en una narrativa universitaria que busca reforzar la conexión emocional y cognitiva del estudiante con el contenido. El backend, implementado con FastAPI y PostgreSQL, garantiza la escalabilidad y el manejo eficiente de datos, mientras que el módulo docente, construido con Next.js, facilita la gestión académica. La validación técnica y pedagógica demostró mejoras en la motivación y comprensión conceptual de los usuarios. Este trabajo contribuye al campo de la tecnología educativa mediante la integración coherente de fundamentos teóricos y herramientas tecnológicas que favorecen un aprendizaje personalizado, significativo y sostenible.

\textbf{Palabras Claves: } videojuegos educativos, gamificación, aprendizaje adaptativo, programación, arquitectura de software, plataformas educativas.

\newpage

\chapter*{Abstract}

This research proposes the design and implementation of an educational platform based on adaptive video games for programming learning. Addressing the limitations of traditional methodologies, the study presents a software architecture that combines gamification, constructivist learning, and adaptive learning principles to enhance motivation and learning effectiveness. The solution consists of an educational video game and a web environment that enables teachers to monitor student progress. The game, developed with the Godot 4 engine, integrates problem-solving mechanics using visual code blocks, embedded in a university-themed narrative that fosters emotional and cognitive engagement. The backend, implemented with FastAPI and PostgreSQL, ensures scalability and efficient data management, while the teacher’s module, built with Next.js, supports academic management. Technical and pedagogical validation revealed improvements in students’ motivation and conceptual understanding. This work contributes to the field of educational technology by coherently integrating theoretical foundations and technological tools to promote personalized, meaningful, and sustainable learning.

\textbf{Keywords: } educational video games, gamification, adaptive learning, programming, software architecture, educational platforms.

% ------------------------------
% Resumen en Español
% ------------------------------
% \begin{abstract}
% El presente documento propone el diseño de una \textbf{plataforma educativa interactiva} orientada al aprendizaje de la programación mediante un \textbf{videojuego educativo adaptativo} y un \textbf{sistema web de gestión académica}. El videojuego permite a los estudiantes practicar conceptos de programación en un entorno lúdico y personalizado, ajustando los niveles y desafíos según el rendimiento individual. El módulo web para profesores facilita la supervisión del progreso de los estudiantes, la generación de reportes y la toma de decisiones pedagógicas basadas en datos. El sistema se sustenta en teorías de aprendizaje basadas en juegos, gamificación, aprendizaje adaptativo y constructivismo, con el objetivo de mejorar la motivación, la retención de conocimientos y la experiencia educativa en general.

% \textbf{Palabras clave:} aprendizaje de programación, videojuego educativo, gamificación, aprendizaje adaptativo, plataforma educativa, gestión académica.
% \end{abstract}

% % ------------------------------
% % Abstract in English
% % ------------------------------
% % \begin{abstract}
% % This document presents the design of an \textbf{interactive educational platform} aimed at teaching programming through an \textbf{adaptive educational game} and a \textbf{web-based academic management system}. The game enables students to practice programming concepts in a playful and personalized environment, adjusting levels and challenges based on individual performance. The web module for teachers allows monitoring student progress, generating reports, and making data-driven pedagogical decisions. The system is grounded in game-based learning, gamification, adaptive learning, and constructivist theories, aiming to enhance motivation, knowledge retention, and the overall educational experience.

% % \textbf{Keywords:} programming learning, educational game, gamification, adaptive learning, educational platform, academic management.
% % \end{abstract}

