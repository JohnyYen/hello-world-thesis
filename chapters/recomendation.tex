\chapter*{Recomendaciones}
\addcontentsline{toc}{chapter}{Recomendaciones}

A partir del proceso de investigación se proponen las siguientes recomendaciones orientadas a mejorar, ampliar y fortalecer futuras versiones del sistema, así como a guiar investigaciones relacionadas. Estas sugerencias buscan aprovechar las fortalezas del prototipo actual y atender los aspectos susceptibles de perfeccionamiento.

\begin{itemize}
    \item \textbf{Ampliar los estudios de usabilidad con una mayor cantidad y diversidad de usuarios.}  
    Se recomienda realizar pruebas con grupos más numerosos, incluyendo estudiantes de diferentes niveles educativos, con el fin de obtener una caracterización más robusta del impacto pedagógico del videojuego y ajustar la experiencia a distintos perfiles de aprendizaje.

    \item \textbf{Integrar nuevos niveles y contextos educativos que aborden conceptos adicionales de programación.}  
    La extensión de los mundos del juego permitiría cubrir otros temas fundamentales como ciclos, funciones, estructuras de selección avanzadas o manipulación de datos, enriqueciendo así el currículo del videojuego.

    \item \textbf{Profundizar en el sistema de adaptación de dificultad.}  
    Se sugiere incorporar modelos más avanzados de análisis del desempeño, explorando técnicas de aprendizaje automático o métricas adicionales que permitan una personalización más precisa y dinámica en función del progreso del estudiante.

    \item \textbf{Optimizar la interfaz de usuario del videojuego.}  
    Se recomienda realizar mejoras en accesibilidad, navegación e interacción, asegurando que el diseño beneficie a estudiantes con diversas necesidades y estilos cognitivos.

    \item \textbf{Incorporar modalidades multijugador o colaborativas.}  
    La colaboración entre pares es un factor que potencia el aprendizaje, por lo que se recomienda explorar funcionalidades que permitan a los estudiantes resolver retos en equipo, discutir soluciones o compartir avances.

    \item \textbf{Implementar herramientas de retroalimentación multimedia.}  
    Se aconseja ampliar el sistema de feedback mediante el uso de audio, animaciones o explicaciones dinámicas que faciliten la comprensión de errores y conceptos clave.

    \item \textbf{Realizar una evaluación pedagógica longitudinal.}  
    Para conocer el impacto real en el aprendizaje, se recomienda evaluar a los estudiantes durante varios meses, comparando resultados antes y después de utilizar el videojuego, y analizando la retención de conocimientos.

    \item \textbf{Optimizar el rendimiento en dispositivos de gama baja.}  
    Debido a la diversidad de equipos utilizados por estudiantes, es recomendable realizar ajustes de rendimiento, compresión de recursos y simplificación gráfica opcional sin comprometer la experiencia educativa.

    \item \textbf{Incorporar un sistema avanzado de logros y recompensas.}  
    Un esquema más elaborado de recompensas podría fortalecer la motivación intrínseca y extrínseca de los estudiantes, ayudando a mantener su interés a lo largo de los niveles.

\end{itemize}

Estas recomendaciones buscan guiar el crecimiento futuro del proyecto y servir como referencia para investigadores, desarrolladores y docentes interesados en el uso de videojuegos educativos como herramienta de aprendizaje.
